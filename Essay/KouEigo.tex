\documentclass[12pt,a4paper]{article}
\usepackage{setspace}
\doublespacing
\usepackage{amsmath}
\usepackage{amsfonts}
\usepackage{amssymb}
\usepackage[left=2cm,right=2cm,top=2cm,bottom=2cm]{geometry}

\title{This is the title}
\author{B4TB2082 Huy Nguyen}
\date{August 5, 2016}

\begin{document}
\maketitle{}
From 9 to 15 March 2016, 20 years after the remarkable victory of IBM's Deep Blue over world chess champion Garry Kasparov, AlphaGo, a computer program developed by Google, won 4-1 in a five-game Go match versus the legendary Korean Go player Lee Sedol. While Deep Blue used a brute force approach, which is basically pre-calculating all the possible outcomes for every move and selecting the most reasonable one, AlphaGo can actually "think" and adjust as the game progresses. The most astonishing thing is AlphaGo had self-taught intuition, creation, and strategic thinking from more than 30 million Go games before the matches and was able to learn from its mistakes. Whether this sets a new milestone in technology development or not, the fact that a program can actually learn, think, and overwhelm humans in a complex game has raised many concerns about advances in artificial intelligence (AI), such as how it could annihilate tons of jobs or even put humans' existence at risk, like in the Terminator movie. Although AI has had a huge impact on our lives, it is also important to be aware of the potential threats. Those are not just rational fears among people, but rather genuine and imminent.

"Advancements in robotics and AI could put almost half of the world's population out of jobs in the coming 30 years," warned Moche Vardi (2016), professor of computational engineering at Rice University. He casted the future picture that will fall upon us: we are reaching the stage that machines will be capable of outperforming humans at almost any task. In fact, as reported by senior officials of Ernst  Young (EY) in 2016, recruitment needs for university graduates in the audit and accounting sector is predicted to decrease to 50\% by 2020 due to competition from AI. Companies will tend not to hire people to do simple tasks, but rather computer scientists that are able to develop algorithms to replace humans in those processes. According to a research done by Citibank and researchers at the University of Oxford's policy school, the Oxford Martin School, 35\% jobs in the UK are potentially being replaced by machines, while in China that figure is 77\% and the world average is 57\%. This will increase the unemployment rate, leading to serious problems like deepening social inequality and causing major crises around the world.

Besides this, the application of AI in military is also controversial. In particular, AI can be exploited to automatically control military arsenals. Unmanned systems can help countries to fight without the loss of soldiers and precise calculations will help to increase chances for victory. For example, in a 2015 Observer article, Michael Sainato states that the US Department of Defense is spending millions on AI technologies, such as an AI that can predict the military strategies of Islamic extremists or an agile version of a six-foot tall, 320-lb humanoid robot that is able to run freely in the woods originally built by a robot company named Boston Dynamics. However, military AI exploitation can lead to a global arms race with obvious risks of enormous destruction and could be seriously detrimental to global economy. In 2015, more than 1000 leading researchers and experts, including professor Stephen Hawking, Apple co-founder Steve Wozniak, Tesla's Elon Musk, have signed a letter calling for a ban on "offensive autonomous weapons". It warns that the "military artificial intelligence arm race" would turn autonomous weapons into the tomorrow's Kalashnikovs, make the battle more likely to happen and can cause a great loss of human life. In other words, the states are high since the military AI systems could easily lead to disastrous consequences and no one can make sure it will be in control once the race among nations starts.

Last but not least, the biggest concern about AI is its limits. According to the recent surveys (October 2015) done by the website AIimpacts.org with experts in AI or related areas as the participants, there will be a 50\% chance of human-level AI range between 2035 and 2050, and 90\% in 2075. But will AI just go from specializing in one narrow task to human intellectually capableness and stop there? The Global Challenges Foudation, which works to raise awareness of the Global Catastrophic Risks, has listed AI in the "risks that threaten human civilization" list, regarding the fear of Artificial Superintelligence, the kind of AI that's much more intelligent than any human. As Tim Urban, one of the Intenet's most popular writers, pointed out in his blog, the intelligence gap among biological creatures is small. We humans can outperform a chimp because our brains are able to execute various sophisticated cognitive modules that carry out actions, like abstract reasoning, rational planning, or complex linguistic behaviours. That resulted from small differences in cognitive abilities between us and chimps, and the same goes between chimps and less intelligent species. So if there will ever be an Artificial Superintelligence (ASI) that is better than us at thinking, we may never be able to comprehend the thing that it can do, like the chimp can never grasp concepts of the human world. The existence of such extreme intelligences is a huge risk to humanity. ASI may or may not send humans to extinction, but ASI definitely make us not the smartest species on Earth. Therefore, we may not be able to "run" this planet like the way we are doing or, even worse, we would be manipulated. This may sound fictional, but the fact that AI, which already can perform calculations much faster and much more precise than humans, can now learn and think with incredible speed and efficiency is an indisputable proof that this idea is not only in fiction novels anymore.

AI is one of the most brilliant inventions of humanity. The development of AI is driven by the development of human wisdom. Yet AI can also be a severe danger for many reasons (for example, it can increase unemployment rate, military AI is a huge risk for humanity and ASI threaten humans' dominant position). Nevertheless, as a matter of fact, humans will always use their inventions in good as well as bad purposes. The important point is that AI development must be planned to be sustainable and safe, not only to best promote its impact in the social life, but also to push human civilization forward.
\end{document}